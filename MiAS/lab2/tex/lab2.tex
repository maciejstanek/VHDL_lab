\documentclass[13pt, a4paper, twoside]{mwart}
\usepackage[a4paper]{geometry}
\geometry{left=3cm}
\geometry{right=1.5cm}
\geometry{top=2cm}
\geometry{bottom=1.5cm}
\usepackage[pdftex]{graphicx}
\graphicspath{{img/}}

\usepackage[utf8]{inputenc}
\usepackage{polski}
\usepackage[polish]{babel}
\usepackage{tabularx}
\usepackage{datetime}
\usepackage{listings}
\usepackage{courier}
\lstset{basicstyle=\footnotesize}

\newcommand{\coursename}{Modelowanie i Analiza Systemów}
\newcommand{\labnumber}{2}
\newcommand{\labname}{Testbench układu decymacyjnego}
\newcommand{\studentname}{Maciej Stanek}
\newcommand{\studentnumber}{122352}
\newdate{labdate}{16}{3}{2018}
\newdate{labreportdate}{30}{3}{2018}

\usepackage{fancyhdr}
\pagestyle{fancy}
\fancyhead[RO,LE]{\thepage}
\fancyhead[LO]{\textbf{LAB\#\labnumber} \labname}
\fancyhead[RE]{\coursename}
\fancyfoot{}

\usepackage{xcolor}
\usepackage{framed}
\colorlet{shadecolor}{gray!10}
\newcounter{taskcounter}
\newcommand{\task}[1]{
  \stepcounter{taskcounter}
  \vspace{0.2cm}
  \begin{shaded}
    \noindent\textbf{Zadanie \thetaskcounter:} \textit{#1}%
  \end{shaded}
  \vspace{0.2cm}}

\begin{document}

\begin{center}
  \textbf{\LARGE{Sprawozdanie z laboratorium}}
\end{center}

\noindent
\begin{tabularx}{\linewidth}{rX}
  \textbf{Przedmiot} & \coursename \\
  \textbf{Temat laboratorium} & \labname \\
  \textbf{Numer laboratorium} & \labnumber \\
  \textbf{Imię i nazwisko} & \studentname \\
  \textbf{Numer indeksu} & \studentnumber \\
  \textbf{Data wykonania} & \displaydate{labdate} \\
  \textbf{Data sprawozdania} & \displaydate{labreportdate} \\
\end{tabularx}

\vspace{0.3cm}
\noindent\hrulefill

%%%%%%%%%%%%%%%%%%%%%%%%%%%%%%%%%%%%%%%%%%%%%%%%%%%%%%%%%%%%%%%%%%%%%%%%%%%%%%%

\task{Opis koncepcji układu decymacyjnego i wyjaśnienie jego działania.}

\task{Opis koncepcji działanie całego układu testbenchu – zależności czasowe taktowania.}

\task{Wyjaśnienie znaczenia/funkcji wszystkich portów (we/wy) oraz parametrów (generic). W kodzie należy nadać wartości domyślne (default) wszystkim parametrom.}

\task{Kody źródłowe VHDL – Testbenchu – bogato udokumentowany komentarzami.}

\task{Kod źródłowy układu decymacyjnego - bogato udokumentowany komentarzami.}

\task{Plik wejściowy (z opisem sposobu jego generacji i wyjaśnienie co jest w nim zakodowane i z jakimi parametrami OSR).}

\task{Plik wyjściowy z wyjaśnieniem zawartości.}

\task{Zrzut ekranu przedstawiający działania układu, który zawiera co najmniej następujące przebiegi (sygnały):
  \begin{itemize}
    \item Zegar bazowy (CLK),
    \item Bitstream wejściowy (Data\_In) --- przed decymacją,
    \item Ciąg wyjściowy liczb całkowitych reprezentujący sygnał wejściowy (Data\_Out).
  \end{itemize}}

%\lstinputlisting[
%  language=VHDL,
%  caption={Implementacja generatora z zadania pierwszego wraz z pustą architekturą}
%  ]{../lab1/gen.vhd}
%
%\lstinputlisting[
%  language=tcl,
%  caption={Skrypt testujący generator z zadania drugiego}
%  ]{../lab1/gen.do}
%
%\begin{figure}[h]
%	\centering
%	\includegraphics[width=\linewidth]{1.png}
%	\caption{Wyniki symulacji generatorów z zadania pierwszego i drugiego.}
%\end{figure}
%
%\begin{figure}[h]
%	\centering
%	\includegraphics[width=0.4\linewidth]{karnaugh.png}
%	\caption{Tablica Karnaugh dla wyjść EQ i GE}
%\end{figure}
%
%\begin{figure}[h]
%	\centering
%	\includegraphics[width=0.4\linewidth]{geeq.png}
%	\caption{Równania wyjść EQ i GE wyznaczone na podstawie tablicy Karnaugh}
%\end{figure}
%
%\lstinputlisting[
%  language=VHDL,
%  caption={Implementacja sumatora w architekturach behawioralnej i strukturalnej}
%  ]{../lab1/uklad_kombinacyjny_porownujacy_dwie_liczby_dwubitowe.vhd}
%
%\lstinputlisting[
%  language=tcl,
%  caption={Skrypt testujący sumator z zadania trzeciego}
%  ]{../lab1/uklad_kombinacyjny_porownujacy_dwie_liczby_dwubitowe.do}
%
%\begin{figure}[h]
%	\centering
%	\includegraphics[width=\linewidth]{compare_structural.png}
%  \caption{Wynik działania architektury strukturalnej}
%\end{figure}
%
%\begin{figure}[h]
%	\centering
%	\includegraphics[width=\linewidth]{compare_behavioral.png}
%  \caption{Wynik działania architektury behawioralnej}
%\end{figure}

%%%%%%%%%%%%%%%%%%%%%%%%%%%%%%%%%%%%%%%%%%%%%%%%%%%%%%%%%%%%%%%%%%%%%%%%%%%%%%%

\end{document}
